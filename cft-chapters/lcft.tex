\chapter{Local class field theory}\llabel{lcft}
We now prove the main theorems of class field theory using cohomology. Throughout this chapter, $K$, $L$, etc. will denote nonarchimedean local fields, unless specified otherwise.\footnote{Local class field theory for $\R$ and $\C$ is trivial and left to the reader. (The only nontrivial field extension is $\C/\R$.)} The main steps are the following.
\begin{enumerate}
\item Construct the invariant map $H^2(K^{\text{ur}}/K)\to \Q/\Z$. (Proposition~\ref{invariant-map})
\begin{enumerate}
\item
Show that $H^2(G(K\ur/K),U_{K\ur})=0$. (Theorem~\ref{cohomology-units-trivial})
\item
From the decomposition $K\urt=U_{K\ur}\times \Z$ and step 1,  %with the projection $v_{K\ur}:K\urt\to \Z$, 
we get $H^2(G(K\ur/K),K^{\text{ur}\times})\cong H^2(G(K\ur/K),\Z)$. (Note the projection $K\urt\to \Z$ is the valuation map $v_{K\ur}$.) Relate $H^2(G(K\ur/K),\Z)$ to $\Q/\Z$ using the long exact sequence in cohomology associated to $0\to \Z\to \Q\to \Q/\Z\to 0$.
\end{enumerate}
\item Now show that there is an isomorphism $\Br_K:=H^2(\ol K/K)\cong H^2(K^{\text{ur}}/K)$ (Theorem~\ref{hkur}). Thus we can restrict attention to unramified extensions of $K$ and use step 1. Unramified extensions are easier to deal with! 
There are two approaches:
\begin{enumerate}
\item
By Theorem~\ref{brauer2} there is an exact sequence
\[
0\to H^2(K\ur/K)\to \Br_K\to \Br_{K\ur}.
\]
Show that $\Br_{K\ur}=0$ by considering central simple algebras over local fields.
%interpreting it in terms of norm groups. (This follows Serre in~\cite{Se79}.)
\item
Study the cohomology of $U_L$ when $L/K$ is cyclic to conclude that the Herbrand quotient $h(U_L)$ is 1. From this get $h(L^{\times})=[L:K]$. From this calculation and Hilbert's Theorem 90~(\ref{h90}), compute\footnote{This is the input for {\it abstract class field theory} according to Neukirch~\cite{Ne99}.}
%(G(L/K),U_{L})=0$ when $L/K$ is cyclic using a filtration of $U_L$. 
\begin{align*}
|H^1(L/K)|&=1,\\
|H^2(L/K)|&=[L:K].
%\cong \Z/[L:K]\Z
\end{align*}
Conclude that $H^2(L/K)$ is cyclic of order $[L:K]$ and hence included in $H^2(K\ur/K)$, for any finite $L/K$.
%H^2(L/K)\cong \Z/[L:K]\Z$ first when $G(L/K)$ is cyclic and then in the general case. Conclude that $H^2(L/K)\subeq H^2(K\ur/K)$. (Serre, in Cassels-Frohlich) %, even when $L$ is ramified. Do this by computing $H^2(L/K)=[L:K]$; show that $H^2(G(K\ur/K),U_{K\ur})
\end{enumerate}
\item Combining the first two steps, we get the invariant map $\inv_K:\Br_K\to \Q/\Z$. Show that this is compatible with restriction and hence that $(G(\ol K/K),\ol K)$ is a {\it class formation}. 
Note $\inv_K$ restricts to $H^2(L/K)\to \rc{[L:K]}\Z$; supposing its image is generated by $u_{L/K}$, Tate's Theorem~\ref{tate-thm} gives an isomorphism
\[
\xymatrix{
H_T^{-2}(G(L/K),\Z)\ar[r]^-{\bullet \cup u_{L/K}}_{\cong}\ar@{=}[d] & H_T^0(G,L^{\times})\ar@{=}[d]\\
G(L/K)^\text{ab}& K^{\times}/\nm_{L/K}(L^{\times})}
\]
that sends $\Frob_{L/K}$ to $[\pi]$ when $L/K$ is unramified. Taking a direct limit, we get a map $K^{\times}\to G(K^{\text{ab}}/K)$. 
Note we only get a map from $G^{\text{ab}}$ (norm limitation).
\item
Study the Hilbert symbol to prove the existence theorem (See Sections~\ref{sec:hilbert-symbol}--\ref{sec:local-existence}).
%\item First we suppose $L/K$ is a finite unramified extension. Show that the (co)homology of the units is trivial (Theorem~\ref{homology-units-trivial}).
%\[
%H_T^r(G,U_L)=0.
%\]
%\item Using the short exact sequence
%\[
%0\to U_L\to L^{\times} \xra{\ord_L}\Z\to 0,
%\]
%and $H_T^r(G,U_L)=0$, we get (by ?) that $H^2(L/K)\cong H^2(G,\Z)$. We relate the latter to $\Q/\Z$, to and construct the invariant map
%\[
%\inv_{L/K}:H^2(L/K)\to \Q/\Z.
%\]
%\item Use Tate's Theorem~\ref{tate-thm} to obtain an isomorphism
%%Define the following map using the cup product.
%\[
%\xymatrix{
%H_T^{-2}(G,\Z)\ar[r]^{\bullet \cup u_{L/K}}_{\cong}\ar@{=}[d] & H_T^0(G,L^{\times})\ar@{=}[d]\\
%G& K^{\times}/\nm(L^{\times})}
%\]
%sending $\si$ to $[\pi]$.
%\item Extend the argument to ramified extensions to obtain the invariant map
%\[
%\inv_{K^{\text{al}}/K}:H^2(K^{\text{al}}/K)\to \Q/\Z
%\]
%compatible with the $\inv_{K/L}$. We have $H^2(K^{\text{al}}/K)$ is a cyclic group; let $u_{L/K}$ be its generator. 
%\item Construct the local Artin map by using Tate's Theorem to get
%\[
%\xymatrix{
%H^{-2}(G,L^{\times})\ar[r]^{\cong}\ar@{=}[d] & H^0(G,L^{\times})\ar@{=}[d]\\
%G^{\text{ab}}& K^{\times}/\nm(L^{\times}).}
%\]
\end{enumerate}
%It is a somewhat onerous task to trace through all the maps to find out what the final map actually is!
Unfortunately it is quite difficult to trace through the maps to find out what the Artin map actually is---for this Lubin-Tate Theory is better. %(Chapter~\ref{lubin-tate}) is much better.
\section{Cohomology of the units}
\index{cohomology of units}
For an unramified extension, the cohomology of the units is trivial.
\begin{thm}[Cohomology of units]\llabel{cohomology-units-trivial}
Suppose $L/K$ is a finite unramified extension of local fields with Galois group $G$. Let $U_L$ be the group of units of $L$. Then 
\[
H_T^r(G,U_L)=1
\]
for any $r$. Hence $H^n(G(K\ur/K),U_{K\ur})=0$ for $n>0$.
\end{thm}
\begin{proof}
We will show that
\[
H_T^1(G,U_L)=H_T^0(G,U_L)=1.
\]
Then it follows from Proposition~\ref{iso+2} that all the Tate groups are trivial. The second part follows from taking the direct limit. %The first we will show by direct computation and Hilbert's Theorem 90; for the second we will show $\nm_{L/K}:U_L\to U_K$ is surjective (Theorem~\ref{unramified-norm-surjective}) by analyzing a filtration on the $U_L$ and $U_K$ and using successive approximations.

We have
\beq{eq:L=ULxZ}
L^{\times}=U_L\times \pi^{\Z}\cong U_L\times \Z
\eeq
where $\pi$ is a uniformizer for $L$. Since $L/K$ is unramified, we can choose $\pi \in K$. Then $G$ acts trivially on $\pi$, so acts trivially on $\Z$ in the decomposition above. Thus~(\ref{eq:L=ULxZ}) gives a decomposition of $L^{\times}$ as a $G$-module (not just as a group). We have by Hilbert's Theorem 90 (Theorem~\ref{h90}) and the fact that cohomology respects products (Proposition~\ref{cohom-preserve-prod}) that
\[
0= H^1(G,L^{\times})=H^1(G,U_L)\times H^1(G,\Z).
\]
Hence $H^1(G,U_L)=1$.

It remains to show $H_T^0(G,U_L)=1$. To do this, let $\mm$ be the maximal ideal of $L$, $U_L^{(m)}:=1+\mm^n$, and consider the filtration
\[
U_K^{(0)}:=U_K\supset U_K^{(1)}\supset U_K^{(2)}\supset \cdots.
\]
Proposition~\ref{units-filtration-2} and~\ref{cohom-finite-fields} below show that each quotient has trivial cohomology:
\[
H_T^0(G,U_L^{(i)}/U_L^{(i+1)})=1.
\]
Then Lemma~\ref{filtration0-h} gives that $H_T^0(G,U_L)=1$, as needed.
\end{proof}
\begin{pr}\llabel{units-filtration-2}
Let $K$ be a complete field with discrete valuation, $\mm$ be the associated maximal ideal, and $U_K^{(m)}:=1+\mm^m$. Then we have isomorphisms
\begin{align*}
U_K/U_K^{(1)}&\xra{\cong} k^{\times}&
U_K^{(m)}/U_K^{(m+1)}&\xra{\cong} k^+\\
u&\mapsto u\pmod{\mm}&
1+a\pi^m&\mapsto a\pmod{\mm}
\end{align*}
that preserve Galois action.
\end{pr}
\begin{proof}
This is Proposition~\ref{units-filtration}.
\end{proof}
\begin{pr}\llabel{cohom-finite-fields}
Let $l/k$ be an extension of finite fields and $G:=G(l/k)$. Then
\begin{align*}
H_T^r(G,l^{\times})&=\{1\}\\
H_T^r(G,l^{+})&=\{0\}
\end{align*}
for all $r\in \Z$. Moreover, the maps $\nm_{l/k}:l\to k$ and $\tr_{l/k}:l\to k$ are surjective.
\end{pr}
\begin{proof}
By Hilbert's Theorem 90 (\ref{h90}), $H^1(G,l^{\times})=0$. Since $G$ is cyclic and $l$ is finite, by Proposition~\ref{herbrand-1}, $h(l^{\times})=1$, giving $H^2(G,l^{\times})=0$. Again since $G$ is cyclic, by Theorem~\ref{iso+2}, all the Tate groups are 0.

From Theorem~\ref{h+0}, $H_T^r(G,l^+)=0$ for $r\ge 0$. %By Theorem~\ref{iso+2}, all the Tate groups are 0.

For the second statement, just note
\begin{align*}
\{1\}&=H_T^0(G,l^{\times})=(l^{\times})^G/N_G(l^{\times})
=k^{\times}/\nm_{l/k}(l^{\times})\\
\{0\}&=H_T^0(G,l^{+})=l^G/N_G(l)
=k/\tr_{l/k}(l).\qedhere
\end{align*}
\end{proof}
\begin{lem}\llabel{filtration0-h}
Let $G$ be a finite group and $M$ be a $G$-module. Let
\[
M=M^0\supeq M^1\supeq\cdots 
\]
be a decreasing sequence of $G$-submodules and suppose $M=\varprojlim M/M^i$ (i.e. $M$ is complete with respect to this filtration). If $H^q(G,M^i/M^{i+1})=0$ for all $i$, then $H^q(G,M)=0$.
\end{lem}
\begin{proof}
Let $f$ be a $q$-cocycle of $M$. 
Since $H^q(G,M/M^1)=0$, the long exact sequence of $0\to M^1\to M\to M/M^1$ gives $H^q(G,M^1)\tra H^q(G,M)$ and we can write $f=g_0+f_1$, where $g_0=\de h_0$ is a coboundary in $M$ and $f_1$ is a $q$-cocycle in $M^1$. Given $f_n\in H^q(G,M^n)$, we can write
\[
f_n=\de h_{n}+f_{n+1}
\] 
where $h_{n}$ is a $(q-1)$-cocycle of $M^n$ and $f_{n+1}$ is a $q$-cocycle of $M^{n+1}$. Then
\[
f=\de(h_1+h_2+\cdots ),
\]
the infinite series being defined in $H^{q-1}(G,M)$ since $h_n$ is a cochain with values in $M^n$, and $M$ is complete with respect to this filtration.
%$M=\varprojlim M/M^i$ gives $H^{q-1}(G,M)=\varprojlim H^{q-1}(G,M/M^i)$. %(Proposition~\ref{group-hom-cohom}).
\end{proof}
%\begin{thm}\llabel{unramified-norm-surjective}
%Let $L/K$ be an unramified extension of local fields. Then 
%\[
%\nm_{L/K}:U_L\tra U_K
%\]
%is surjective.
%\end{thm}
%\begin{proof}
%The following commute (examine the maps in Proposition~\ref{units-filtration})
%\begin{equation}
%\xymatrix{
%U_L \ar@{->>}[r] \ar[d]^{\nm_{L/K}}& l^{\times} \ar@{->>}[d]^{\nm_{l/k}}\\
%U_K\ar@{->>}[r] & k^{\times}}
%\quad
%\xymatrix{
%U_L^{(m)} \ar@{->>}[r] \ar[d]^{\nm_{L/K}}& l^{+} \ar@{->>}[d]^{\tr_{l/k}}\\
%U_K^{(m)} \ar@{->>}[r] & k^{+}.}
%\end{equation}
%Hence given $u\in U_K$ we can choose $v_0$ such that $u\equiv \nm_{L/K}(v_0)\pmod{\mm}$, i.e.
%\[
%\frac{u}{\nm(v_0)}\in U_K^{(1)}.
%\]
%Supposing $u_j:=\frac{u}{\nm(v_0\cdots v_j)}\in U_K^{(j+1)}$, from the right-hand square we can choose $v_{j+1}\in U_L^{(j+1)}$ so that 
%\[\frac{u'}{\nm_{L/K}(v_{j+1})}
%=\frac{u}{\nm_{L/K}(v_0\cdots v_{j+1})}\in U_K^{(j+2)}.\]
%Take $v=\prod_{n=0}^{\iy} v_n$; then $\nm_{L/K}(v)=u$. 
%\end{proof}
This proves Theorem~\ref{cohomology-units-trivial}. We record the following corollary, for easy reference.
\begin{cor}\llabel{thm:local-nm-surj}
Suppose $L/K$ is a finite extension of local fields. Then
\[
U_K\subeq \nm_{L/K}U_L.
\]
\end{cor}
\begin{proof}
%Since $\nm_{L/K}(a)$ is a unit in $K$ iff $a$ is a unit in $L$, it suffices to show 
%It suffices to show 
If $L/K$ is Galois, then this follows since by Theorem~\ref{cohomology-units-trivial}
\[
U_K/\nm_{L/K}U_L=H_T^0(G(L/K),U_L)=\{1\}
\]
so the norm map $U_L\to U_K$ is surjective.

For general extensions $L/K$, consider the Galois closure and use transitivity of norms.
\end{proof}
\section{The invariant map}
\subsection{Defining the invariant maps}
\index{invariant map}
\begin{pr}\llabel{invariant-map}
For any finite unramified Galois extension of local fields $L/K$ there is a canonical isomorphism
\[
\inv_{L/K}:H^2(L/K)\xra{\cong} \rc{[L:K]}\Z/\Z.
\]
Taking the direct limit gives an injective map
\[
\inv_{K\ur/K}:H^2(K\ur/K)%\xra{\cong} 
\to\Q/\Z.
\]
\end{pr}
\begin{proof}
Consider the short exact sequence
\[
1\to U_L\to L^{\times} \xra{v_L}\Z\to 0.
\]
Since $H_T^n(G,U_L)=0$ for all $n$ by Theorem~\ref{cohomology-units-trivial}, taking the long exact sequence gives
\[
\cancelto{0}{H^2(G,U_L)}
\to H^2(L/K)\xra{\cong} H^2(G,\Z) \to \cancelto{0}{H^3(G,U_L)}.
\]
We relate $H^2(G,\Z)$ to a lower cohomology group by considering the short exact sequence
\[
0\to \Z\to \Q\to \Q/\Z\to 0.
\]
Note $H^n(G,\Q)$ is torsion for any $n>0$ by Corollary~\ref{hn-torsion}. Since $\Q$ is a divisible group, so is $H^n(G,\Q)$, by looking at the description of $H^n$ in terms of cocycles (Section~\ref{sec:bar-res}). 
Hence 
$H^n(G,\Q)=0$ for any $n>0$.
Taking the long exact sequence of the above we get
\[
\cancelto{0}{H^1(G,\Q)} \to H^1(G,\Q/\Z) \xra{\cong}
H^2(G,\Z) \to \cancelto{0}{H^2(G,\Q)}.
\]
Thus we get a map
\beq{eq:inv-unram}
\inv_{L/K}:\quad 
H^2(L/K)\xra{\cong} H^2(G,\Z) \xleftarrow{\cong} H^1(G,\Q/\Z) \stackrel{\ref{h1-is-hom}}{\cong} \Hom(G,\Q/\Z) %\xra{\cong}
\xra{\cong}\rc{[L:K]}\Z/\Z.
\eeq
where the last is defined by taking the Frobenius element $\si$ of $G$ and mapping $f\mapsto f(\si)$. (Note $G$ is cyclic and $\si$ generates $G$; the Frobenius is a canonical choice.)

Now define $\inv_{K\ur/K}=\varinjlim_{L/K\text{ finite Galois unramified}} \inv_{L/K}$, taking the direct limit under inflation. Since inflation is functorial, the first two maps in~(\ref{eq:inv-unram}) commute with it. Identifying $H^1(G,\Q/\Z)\cong \Hom(G,\Q/\Z)$, inflation sends a map $G(L/K)\to \Q/\Z$ to $G(M/K)\tra G(L/K)\to \Q/\Z$. Moreover, $\Frob_{L/K}$ is the projection of $\Frob_{M/K}$ to $G(L/K)$. Hence $\Inf_{M/L}$ commutes with the inclusion map $\rc{[L:K]}\Z/\Z\hra \rc{[M:K]}\Z/\Z$, and the $\inv_{L/K}$ form a compatible system under inflation.
\end{proof}
\begin{rem}
Let $K$ be any nonarchimedean complete field (not necessarily local) with residue field $k$. Then %we have the split exact sequence
\[
%1\to H^n(l/k)\to H^n(L/K)\to H^n(G(K\ur/K),\Q/\Z)
%%\cong\Hom^1(G(K\ur/K),\Q/\Z)
%\to 1.
H^n(L/K)=H^n(l/k)\times H^n(G(L/K),\Q/\Z).
\]
%splitting from uniformizer
Indeed, Proposition~\ref{units-filtration-2} and Theorem~\ref{h+0} still give
\[
H_T^r(G,U_L^{(i)}/U_L^{(i+1)})\cong H_T^r(G,l^{+})=0
\]
for $i\ge 1$. This gives $H_T^r(G,U_L^{(1)})=0$ by Lemma~\ref{filtration0-h}. 
From the long exact sequence associated to
\[
1\to U_L^{(1)}\to U_L\to U_L/U_L^{(1)}\cong l^{\times}\to 1
\]
we get
%Then~(\ref{eq:L=ULxZ}) gives 
\[
H^n(L/K)\cong H^n(G,U_L)\times H^n(G,\Z)=H^n(G,l^{\times})\times H^n(G,\Z).
\]
In the case of a local field, $l$ was finite so $H^n(G,l^{\times})=1$.
\end{rem}
%
\subsection{Compatibility of the invariant maps}
We show that the invariant maps are compatible, in the following sense.
\begin{thm}\llabel{thm:inv-compatible}
Let $L/K$ be a Galois extension of local fields, and $n=[L:K]$. Then
\[
\inv_{K\ur/L}\circ \Res_{K/L}=n\inv_{K\ur /K}
\]
\end{thm}
\begin{proof}
To do this we have to unravel all those steps we took to define $\inv_{K\ur/K}\ldots$ We first prove this for two special cases.
\begin{enumerate}
\item $L/K$ is unramified. Let $G=G(K\ur/K)$ and $S=G(K\ur/L)$. We claim the following commutes.
\[
\xymatrix{
 H^2(K\ur/K) \ar[r]\ar[d]^{\Res} & H^2(G,\Z)\ar[d]^{\Res} & H^1(G, \Q/\Z)\ar[l]\ar[r]^{\ga} \ar[d]^{\Res} & \Q/\Z\ar[d]^n\\
 H^2(K\ur /L)\ar[r] & H^2(S,\Z) & H^1(S, \Q/\Z) \ar[l]\ar[r]^{\ga} & \Q/\Z.
}
\] 
For the squares involving $\Res$, this follows from naturality of $\Res$. For the last square, identify $H^1(G,\Q/\Z)=\Hom(G,\Q/\Z)$; $\Res$ becomes simply restriction of homomorphisms.
Recall that $\ga$ was defined taking the Frobenius $\Frob(K\ur/K)\in G(K\ur/K)$ and sending $f\in H^1(G,\Q/\Z)=\Hom(G,\Q/\Z)$ to $f(\si)$, and we have
\[
\Frob_{K\ur/K}^{n}=\Frob_{K\ur /L}
\]
by Proposition~\ref{pr:frob-base-ext}.
\item $L/K$ is totally ramified. 
Note that $G=G(K\ur/K)=G(K\ur L/L)=G(L\ur/L)$ in this case, from the description of $K\ur$ in Theorem~\ref{unram-ram}. 
We show the following commutes:
\[
\xymatrix{
H^2(K\ur/K) \ar[r]\ar[d]^{\Res} & H^2(G,\Z)\ar[d]^{n} & H^1(G, \Q/\Z)\ar[l]\ar[r]^{\ga} \ar[d]^{n} & \Q/\Z\ar[d]^n\\
H^2(K\ur /L)\ar[r] & H^2(G,\Z) & H^1(G, \Q/\Z) \ar[l]\ar[r]^{\ga} & \Q/\Z.
}
\]
%Here $i$ is the ``change of group" map induced by inclusion $K\urt\hra L\urt$, so the first square commutes. 
The first square commutes by commutativity of
\[
\xymatrix{
K\urt \ar[r]^{v_K}\ar@{^(->}[d] & \Z\ar[d]^n\\
L\urt \ar[r]^{v_L} & \Z.
}
\] 
(and of course, naturality of cohomology). Here $v_K$ and $v_L$ are the valuation maps, i.e. the projections $K^{\times}\cong U_K\times \Z\to \Z$ and $L^{\times}\cong U_L\times \Z\to \Z$.
%alternatively we can combine the 2 cases, and work with ramification/residue field degree. See Milne III.1.8.
\end{enumerate}
The general case follows by considering $L/L^{I_{L/K}}$ (totally ramified) and $L^{I_{L/K}}/K$ (unramified). (See Theorem~\ref{decomposition-and-inertia}.)
\end{proof}
\section{$H^2(\ol K/K)\cong H^2(K^{\text{ur}}/K)$}
We prove the following.
\begin{thm}\llabel{hkur}
The inclusion (inflation) map
\[
H^2(K\ur/K)\to H^2(\ol K/K)
\]
is an isomorphism.
\end{thm}
For short we write $H^2(K):=H^2(\ol K/K)$.
\subsection{First proof (Brauer group)}
\begin{proof}[First proof]
By Proposition~\ref{brauer2} there is an exact sequence
\[
0\to H^2(K\ur/K)\to H^2(K)\to H^2(K\ur)= \cancelto{0}{\Br_{K\ur}}.
\]
The last term is zero by Theorem~\ref{brkur0} below. %(Recall this was proved using surjectivity of the norm map OR by calculating the Brauer group.) 
Thus we get $H^2(K\ur/K)\cong H^2(K)$, as needed.
\end{proof}
\begin{thm}\llabel{brkur0}
Let $K$ be a local field. Then $\Br_{K\ur}=0$. 
\end{thm}
%\begin{proof}[Proof 1]
%Use surjectivity of norm (Serre, V.4.7).
%\end{proof}
%\begin{proof}[Proof 2]
%Calculate the Brauer group algebraically (Serre, XII). We use two lemmas.
We need two lemmas.
\begin{lem}\llabel{lem:brkur0-1}
Suppose $D$ is a central division algebra of rank $n^2>1$ over a field $K$, and the residue field $k$ is perfect. Then there exists a  commutative subfield $L$ of $D$ properly containing $K$, unramified over $K$.
\end{lem}
\begin{lem}\llabel{lem:brkur0-2}
Keep the same hypotheses as Lemma~\ref{lem:brkur0-1}. There is a subfield of $D$ of degree $n$ unramified over $K$.
\end{lem}
Note this is a maximal subfield by Corollary~\ref{cor:csa-split}.
\begin{proof}[Proof of Lemma~\ref{lem:brkur0-1}]
Suppose by way of contradiction that every commutative subfield $L$ of $D$ properly containing $K$ is ramified. Then the extension of residue fields $l/k$ must be trivial (see Theorem~\ref{decomposition-and-inertia}). Let $a\in D$ be integral and $\pi\in D$ be a uniformizer for $D$.
(See Proposition~\ref{pr:div-alg-val}.) 
%\footnote{We can define a valuation and a notion of integrality similar to the commutative case. See Serre~\cite{Se79}, p. 182-183 for details.} 
Since $l=k$, there exists $b\in K$ such that $b\equiv a\pmod{\pi}$, and we can write $a=b+\pi b_1$ for some $b_1\in \sO_D$, where $\sO_D$ is the ring of integers in $D$. Iterating this with $b_1$, we find 
\[
a=b+\pi b_1+\cdots +\pi^{n-1} b_{n-1}+\pi^n b_n
\]
where $b_1,\ldots, b_{n-1}\in \sO_K$ and $b_n\in \sO_D$. Thus $a$ is in the closure of $K(\pi)$. But $K(\pi)$ is closed (it is a finite-dimensional vector space over $K$), so $a\in K(\pi)$, i.e. $D=K(\pi)$ and $D$ is commutative, a contradiction.
\end{proof}
\begin{proof}[Proof of Lemma~\ref{lem:brkur0-2}]
%This will follow from the previous lemma after a centralizer calculation. 
Induct on $n$. The case $n=1$ is clear. Let $n\ge 2$. By Lemma~\ref{lem:brkur0-1} there exists a proper unramified extension $K'/K$ inside $D$. Let $D'=C(K')$. Since $D'\subeq D$, $D'$ must be a division algebra (a finite dimensional integral domain must contain inverses). Let its center be $K''$. The maximal commutative subfield of $D'$ then has dimension $\sqrt{[D':K'']}$ over $K''$, or dimension $\sqrt{[D':K'']}[K'':K]=\sqrt{[D':K][K'':K]}$ over $K$. This is at most $n$, since the field is also contained in $D$. But $\sqrt{[D':K][K':K]}=n$ by the double centralizer theorem~\ref{thm:dct-gen}, so we must have $K''=K$. Thus $D'$ is a division algebra with center $K'$.
Its degree over $K'$ is less than $n^2$, so %(Double centralizer theorem~\ref{thm:dct-gen}). 
by the induction hypothesis, $D'$ has a maximal commutative subfield $L$ containing $K'$, of degree $\sqrt{[D':K']}$, and unramified over $K'$, hence over $K$. We calculate
\[
[L:K]=[L:K'][K':K]=\sqrt{[D':K']}[K':K]=\sqrt{[D':K][K':K]}=\sqrt{[D:K]}
\]
where we used Theorem~\ref{thm:dct-gen} in the last step. This finishes the induction step.
\end{proof}
\begin{proof}[Proof of Theorem~\ref{brkur0}]
Suppose $D$ is a central division algebra over $K\ur$ of rank $n^2$. Then lemma 2 furnishes a subfield of $K\ur$ of degree $n$, unramified over $K\ur$. Hence $n=1$, and $D$ is trivial. Thus $\Br_{K\ur}=0$. This proves Theorem~\ref{brkur0} and hence Theorem~\ref{hkur}.
\end{proof}
%\end{proof}cong H^2(K^{\text{ur}}/K)$ (Second approach)}
\subsection{Second proof (Herbrand quotient calculation)}
\subsubsection{Herbrand quotient calculation}
We first need the following lemma.
\begin{lem}\llabel{subgroup-trivial-cohom}
Given a local field $L$, there exists an open subgroup $V$ of $U_L$ with trivial cohomology, i.e. $H^q(G,V)=0$ for all $q$.
\end{lem}
\begin{proof}
%Let $G=G(L/K)$. 
The idea is to compare a multiplicative $G$-module $V$ with an additive $G$-module (more accurately, compare the filtration of $V$), and use the same argument as in Theorem~\ref{h+0}.\footnote{If $\chr(L)=0$ there is a faster proof: Note that $e^x$ is a topological isomorphism from a neighborhood of $0$ in the additive group $L$ to a neighborhood of $1$ in the multiplicative group $\sO_L$. Moreover, it preserves the action of $G$ because the fact that $G$ acts continuously on $L$ gives
\[
e^{\si x}=\su \fc{(\si x)^n}{n!} =\su \fc{\si (x^n)}{n!}=\si e^x.
\]
Now Theorem~\ref{h+0} applies directly.}

By the normal basis theorem, $L^+$ has a normal basis $\set{\si(\al)}{\si\in G}$, i.e. it is free over $K[G]$. Let $A=\sum_{\si\in G} \sO_K\si(\al)$.\footnote{Warning: $A$ is a $\sO_K[G]$-module; we don't know it is an $\sO_L$-module.}
By multiplying $\al$ by a power of $\pi_K$ we may assume that $\al\in \sO_L$. %\footnote{Careful, we're now working with $\sO_L$ instead of $\sO_K$.}
Suppose that
\[
\pi_K^n\sO_L\subeq A\subeq \sO_L.
\]
Let $M=\pi_K^{n+1}A$,  $V=1+M$ and $V^{(i)}=1+\pi_K^{i} M$.
Note that
\[
M\cdot M\subeq \pi_K^{2n+2}A\cdot A\subeq \pi_K\pi_K^{n+1}\pi_K^n\sO_L\subeq
\pi_K\pi_K^{n+1}A\subeq
\pi_KM.
\]
This shows that
\begin{enumerate}
\item
$V$ is a subgroup: Indeed, $(1+M)(1+M)\subeq 1+M+M\cdot M\subeq 1+M$ by the above.
\item
$V^{i}/V^{i+1}\cong A/\pi_K A$ as $G$-modules. Indeed, if $m_1,m_2\in M$, then for some $m_3\in M$, we have
\[
(1+\pi_K^im_1)(1+\pi_K^im_2)= 1+\pi_K^i(m_1+m_2)+\pi_K^{2i} \pi_Km_3
\equiv 1+\pi_K^i(m_1+m_2)\pmod{\pi_K^{i+1}M}.
\]
\end{enumerate}
Hence
\[
H^q(G, V^{(i)}/V^{(i+1)})=H^q(G,M/\pi_KM)=0
\]
for each $q$, since $M/\pi_K M$ is an induced module over $G$ (and has trivial cohomology by Shapiro's Lemma~\ref{shapiro}). (By construction $M/\pi_K M=\Ind^G[(\pi_K^{n+1}\al\sO_K)/(\pi_K^{n+2}\al\sO_K)]$.) Lemma~\ref{filtration0-h} applied to $V$ finishes the proof.
\end{proof}
\begin{pr}\llabel{herbrand-units-l}
Suppose $L/K$ is cyclic of degree $n$. Then
\begin{align*}
h(U_L)&=1.\\
h(L^{\times})&=n.
\end{align*}
\end{pr}
\begin{proof}
Choose $V$ as in Lemma~\ref{subgroup-trivial-cohom}. Since $V$ is open, $U_L/V$ is finite. By Proposition~\ref{herbrand-1}(1), $h(U_L/V)=1$. Hence
\[
h(U_L)=h(V)h(U_L/V)=1.
\]
By Proposition~\ref{herbrand-1}(3), $h(\Z)=|G|=n$. 
Since $L^{\times}=U_L\times \pi_L^{\Z}$ we get %check compatible
\[
h(L^{\times})=h(U_L)h(\Z)=n.
\]
\end{proof}
\begin{thm}[Class field axiom for local class field theory]\llabel{thm:cfa-lcft}
Let $L/K$ be a cyclic extension of degree $n$. Then
\begin{align*}
|H^1(L/K)|&=1\\
|H^2(L/K)|&=n.
\end{align*}
\end{thm}
\begin{proof}
The first follows directly from Hilbert's Theorem 90~(\ref{h90}). For the second, we have $|H^2(L/K)|=h(L^{\times})|H^1(L/K)|=n$ using Proposition~\ref{herbrand-units-l}.
\end{proof}
We want to show that $|H^2(L/K)|=n$ for all Galois extensions $L/K$, and in fact $H^2(L/K)$ is cyclic of order $n$. We proceed in 2 steps.
%We now use this to give a second proof of Theorem~\ref{hkur}. In the course of our proof we will show that for $L/K$ cyclic of degree $n$, $H^2(L/K)$ is actually a cyclic group of order $n$. We already know this for $L/K$ unramified (Theorem~\ref{invariant-map}).
\subsubsection{First inequality}
We show that for all Galois extensions $L/K$, $|H^2(L/K)|\ge [L:K]$. In fact, we show the following.
\begin{lem}\llabel{lem:local-1eq}
Let $L/K$ be a Galois extension of local fields of degree $n$. Then $H^2(L/K)$ contains a subgroup canonically isomorphic to $\rc n\Z/\Z$.
\end{lem}
\begin{proof}
We prove this using Theorem~\ref{thm:inv-compatible}, which relates the invariant maps on $K\ur/K$ and $L\ur/L$. By Theorem~\ref{brauer2}, we have the exact sequence $0\to H^2(L/K)\to H^2(K)\to H^2(L)$. 
Inflation and restriction commute by functoriality of change of group, so we have the commutative diagram with exact columns
\beq{eq:local-1eq}
\xymatrix{
0\ar[d] & 0\ar[d]\\
H^2(L/K)\ar[d] & \ker(\Res)\ar@{.>}[l]\ar[d]\\
H^2(K) \ar[d]^{\Res} & H^2(K\ur/K)\ar@{_(->}[l]_-{\Inf}\ar[d]^{\Res}\\
H^2(L) & H^2(K\ur/L)\ar@{_(->}[l]_-{\Inf}.
}
\eeq
By Theorem~\ref{thm:inv-compatible}, the map $H^2(K\ur/K)\to H^2(L\ur/L)$ corresponds to the multiplication-by-$[L:K]$ map after identifying both sides with a subgroup of $\Q/\Z$ through the respective invariant maps. Hence $\ker(\Res)=\rc{n}\Z/\Z$. The top map exists and is an injection because the other two are (4-lemma). Hence $\rc n\Z/\Z\hra H^2(L/K)$, as needed.
\end{proof}
\subsubsection{Second inequality}
Next we show $|H^2(L/K)|\le [L:K]$, so $|H^2(L/K)|=[L:K]$.
\begin{lem}\llabel{lem:local-2ineq}
Let $L/K$ be a Galois extension of local fields of degree $n$. Then $H^2(L/K)\cong\rc n\Z/\Z$.
\end{lem}
\begin{proof}
We already know that $|H^2(L/K)|=[L:K]$ for $L/K$ cyclic (Theorem~\ref{thm:cfa-lcft}). We prove that $|H^2(L/K)|=[L:K]$ by induction on the degree.

By Corollary~\ref{cor:G(local)=solvable}, $G(L/K)$ is solvable. Thus, if $G(L/K)$ is not cyclic, it has a normal subgroup $G(L/K')$. By Theorem~\ref{brauer2} we have an exact sequence
\[
0\to H^2(K'/K)\to H^2(L/K)\to H^2(L/K')
\]
so
\[
|H^2(L/K)|\le |H^2(K'/K)|\cdot |H^2(L/K')|=[K':K][L:K']=[L:K].
\]
By Lemma~\ref{lem:local-1eq}, equality holds.
\end{proof}
\subsubsection{Finishing the proof}
\begin{proof}[Second proof of Theorem~\ref{hkur}]
Take any element $a\in H^2(\ol K/K)$; it is in $H^2(L/K)$ for some finite Galois $L/K$. The top injection in~(\ref{eq:local-1eq}) is an isomorphism by Lemma~\ref{lem:local-2ineq}, and we get $a\in H^2(K\ur/K)$.
\end{proof}
%Now we can use abstract class field theory!
\index{class formation}
\section{Class formations}
\llabel{sec:class-formations}
The preceding sections show that 
\[(G(\ol K/K),\set{G(L/K)}{L/K\text{ finite Galois}},\ol K)\]
is a {\it class formation}. That is, it satisfies the basic axioms that allow us to obtain the conclusions of class field theory. With the abstraction of class formations, when we prove global class field theory, we only have to verify the axioms and we will get the desired conclusions in the same way as in local class field theory.
\subsection{Class formations in the abstract}
\index{abstract Galois group}
\begin{df}
An \textbf{abstract Galois group} is a group $G$ with a family of subgroups of finite index $\{G_L\}_{L\in X}$ such that
\begin{enumerate}
\item (Closure under intersection) If $L_1,L_2\in X$, then there exists $M$ such that
\[
G_{L_1}\cap G_{L_2}=G_M.
\]
\item (Closure under superset) If $G_L\subeq G'\subeq G$ are subgroups, then $G'=G_{K'}$ for some $K'$.
\item (Closure under conjugation) For every $s\in G$ and $L\in X$ there exists $L'$ so that 
\[sG_Ls^{-1}=G_{L'}.\]
\end{enumerate}
\end{df}
This definition is motivated by the fact that these are the key properties of Galois groups.
\begin{pr}
A topological Galois group $G(\Om/K_0)$ with all its closed subgroups, is an abstract Galois group.
\end{pr}
\begin{proof}
By the fundamental theorem of infinite Galois theory~\ref{ftogt-infinite}, the closed subgroups of $G(\Om/K_0)$ are exactly those in the form $G(\Om/K)$ with $K_0\subeq K\subeq \Om$. The above properties correspond to the following facts from Galois theory.
\begin{enumerate}
\item
$
G(\Om/K)\cap G(\Om/L)=G(\Om/KL).
$
\item The subgroups of $G(\Om/K_0)$ containing $G(\Om/L)$ correspond to intermediate extensions between $K_0$ and $L$.
\item $sG(\Om/K)s^{-1}=G(\Om/sK)$.
\qedhere
\end{enumerate}
\end{proof}
We transfer some terminology about Galois groups to the abstract case.
\begin{df}
Let $(G,\{G_L\}_{L\in X})$ be an abstract Galois group. The elements of $X$ are called fields. The field $K_0$ with $G_{K_0}=G$ is called the basefield. 
For $G_M\subeq G_L$, define $[M:L]$ to be $[G_L:G_M]$; 
we say $M/L$ is a Galois extension if $G_M\trianglelefteq G_L$, and write
\[
G(M/L)=G_L/G_M
\]
(called the ``Galois group" of $M/L$). We say $M/L$ is abelian, etc. if $G(M/L)$ is abelian, etc.

The field $M$ such that $G_{L_1}\cap G_{L_2}=G_{M}$ is called the composite of $L_1$ and $L_2$, and denoted by $L_1L_2$; the field $L'$ such that $sG_Ls^{-1}=G_{L'}$ is denoted by $sL$.
\end{df}
Note every extension $M/L$ is contained in a Galois extension: Since $[G_L:G_M]$ is finite $G_M$ has finitely many conjugates $sG_Ms^{-1}$ in $G_L$; by the axioms $G_{M'}=\bigcap_s sG_Ls^{-1}$ for some $M'$, called the Galois closure of $M/L$.
\index{formation}
\begin{df}
A \textbf{formation} is a triple $(G, \{G_K\}_{K\in X}, A)$ where $(G, \{G_K\}_{K\in X})$ is an abstract Galois group and $A$ is a discrete topological $G$-module (see Definition~\ref{top-g-mod}). Let $A_K:=A^{G_K}$.
%, i.e. $A$ is equipped with a topology and the map
%\begin{align*}
%G\times A&\to G\\
%(g,a)&\mapsto ga
%\end{align*}
%is continuous.

Define the norm $\nm_{L/K}:A_L\to A_K$ by letting $\nm_{L/K}(a)=\prod_{\si\in G(L'/K)/G(L'/K)} \si(a)$ for any $L'$ Galois over $K$.
%\pa{\prod_{a'\in S(a)}a'}^{\fc{[L:K]}{|S(a)|}}$ where $S(a)=\set{a'}{a'=\si a\text{ for some }\si\in G_K}$ (cf. Theorem~\ref{ntr-min-roots}(c)).
\end{df}
For $L/K$ Galois, we define $H^n(L/K):=H^n(G(L/K),A_L)$. We can define inflation, restriction, and corestriction maps in the natural way, with $\Res_{K/L}=\Res_{G_K/G_L}$, and so forth.
\begin{df}\llabel{def:class-formation}
A \textbf{class formation} is a formation $(G, \{G_K\}_{K\in X}, A)$ with a homomorphism $\inv_{L/K}:H^2(L/K)\to \Q/\Z$ for each Galois extension $L/K$, such that the following hold.
\begin{enumerate}
\item
$H^1(L/K)=0$ for every cyclic extension of prime degree.
\item
$\inv_{L/K}$ %is injective on $H^2(K)$ and maps 
is an isomorphism from $H^2(L/K)$ to $\rc{[L:K]}\Z/\Z$. 
\item (Compatibility under inflation) For any finite extension $M/L$, 
\[\inv_{M/K}\circ \Inf_{M/L}=\inv_{L/K}.\]
Hence we can define $\inv_K:\varinjlim_L H^2(L/K)\to \Q/\Z$. (This axiom implies that inflations are injective on $H^2$, so we can think of  $H(K):=\varinjlim_L H^2(L/K)$ as $\bigcup_L H^2(L/K)$.)
%Alternatively, inflations are injective by Proposition~\ref{inflate-restrict} and axiom 1.
%Note inflations are injective on $H^2$ by Proposition~\ref{inflate-restrict} and axiom 1, so we can think of  $H(K):=\varinjlim_L H^2(L/K)$ as $\bigcup_L H^2(L/K)$.)
\item (Compatibility with restriction) For any finite Galois extension $L/K$,
\[
\inv_L\circ \Res_{K/L}=[L:K]\inv_K.
\]
\end{enumerate}
\index{fundamental unit}
Define the \textbf{fundamental unit} of $L/K$ to be
\[
u_{L/K}=\inv_K^{-1}\prc{[L:K]}.
\]
\end{df}
%Axiom 2 says the maps $\inv_E$ are compatible under inflation, so taking $\varinjlim$ we get a homomorphism $\inv_K:H^2(K):=\varinjlim_L H^2(L/K)\to \Q/\Z$. Note inflations are injective on $H^2$, so we can think of the LHS as $\bigcup_L H^2(L/K)$.
%%THIS IS VERY IMPORTANT TO REMEMBER.

\begin{pr}
Assume a formation satisfies axiom 1. Then for every Galois extension $L/K$,
\[
H^1(L/K)=0.
\]
\end{pr}
\begin{proof}
First we show this when $[L:K]$ is a prime power $p^n$. Induct on the degree. The base case is given. Every $p$-group has a subgroup of index $p$, so there is $K\sub K'\sub L$ such that $G(K'/K)$ has order $p$. By the inflation-restriction  exact sequence~\ref{inflate-restrict}, we get
\[
0\to \cancelto 0{H^1(K'/K)}\xra{\Inf} H^1(K/L)\xra{\Res} \cancelto 0{H^1(L/K')};
\]
the first and last terms are 0 by axiom 1 and by the induction hypothesis. So $H^1(K/L)=0$.

For general $L/K$, this shows $H^1(G(L/K)_p,A_L)=0$, so the result follows from Corollary~\ref{cor:all-gp-0}.
\end{proof}
\begin{pr}
Assume a formation satisfies axiom 2. Transferring the action of $\Res$, $\Cor$, and $\Inf$ to the subgroups of $\Q/\Z$, we get the following diagram:
%as acting on $\rc{[K':K]}\Z/\Z\cong H^2(K'/K)$, their actions are given by the following.
\[
\xymatrix{
M\ls{d}_{[M:L]} & & & & & \\
L\ls{d}_{[L:K]} & H^2(M/L)\ar[r]^{\inv_L}
\ar@/_1pc/@{^(->}[d]_{\Cor_{L/K}}
& \rc{[M:L]}\Z/\Z\ar@/_1pc/@{^(->}[d]_{i}&L\ls{d}&&\\
K & H^2(M/K) \ar[r]^{\inv_K} \ar@/_1pc/@{->>}[u]_{\Res_{K/L}} & \rc{[M:K]}\Z/\Z
\ar@/_1pc/@{->>}[u]_{[L:K]}
& K & H^2(L/K)\ar[r]^{\inv_K} \ar@/^1pc/@{^(->}[lll]^{\Inf_{M/L}}
&\rc{[L:K]}\Z/\Z \ar@/^1pc/@{^(->}[lll]^{i}
}
\]
(Note $\Cor_{L/K}\circ \Res_{K/L}=[L:K]$.) 
Moreover (passing to the limit), the following hold.
\begin{enumerate}
\item For every extension $L/K$, 
\[\Res_{K/L}:H^2(K)\tra H^2(L)\]
is surjective.
\item For every extension $L/K$,
\[
\Cor_{L/K}:H^2(L)\hra H^2(K)
\]
is injective, and
\[
\inv_K\circ \Cor_{L/K}=\inv_L.
\]
\item For every $s\in G$, letting $s^*:H^2(K)\to H^2(sK)$, 
\[
\inv_{sK}\circ s^*=\inv_K.
\]
\end{enumerate}
\end{pr}
\begin{proof}
The surjectivity of $\Res_{K/L}$ in the diagram comes directly from the injectivity of $\inv_K$ and $\inv_L\circ \Res_{K/L}=[L:K]\inv_K$.

For the action of $\Cor_{L/K}$, note
\[
\inv_K\circ \Cor_{L/K}\circ\Res_{K/L}=\inv_K\circ[L:K] =\inv_L\circ \Res_{K/L}
\]
where the first follows from Theorem~\ref{corres} and the second from the axiom. Surjectivity of $\Res_{K/L}$ gives $\inv_K\circ \Cor_{L/K}=\inv_L$, as needed.

Items 1 and 2 now follow from taking the direct limit.
%and injectivity of $\inv_L$ gives injectivity of $\Cor_{L/K}$.

For 3, let the basefield be $K_0$; note the map $s^*:H^2(K_0)\to H^2(sK_0)=H^2(K_0)$ is the identity by Proposition~\ref{change-group-conjugation}, so $\inv_{sK_0}\circ s^*=\inv_{K_0}$. For arbitrary $x\in H^2(K)$, by surjectivity of $\Res_{K/L}$ we can write $x=\Res_{K/L}(x_0)$. Since $\Res$ and $s^*$ commute (transport of structure),
\[
\inv_{sK}(s^*x)=\inv_{sK} (s^*\Res_{K_0/K}x_0)
=\inv_{sK} \Res_{sK/sK_0}(s^*x_0)
=[sK:sK_0] \inv_{sK_0}(x_0)=\inv_K(x).
\]
\end{proof}
The reciprocity law follows from the properties of class formations.
\index{reciprocity law}
\begin{thm}[Abstract reciprocity law]\llabel{thm:abstract-reciprocity}
Let $(G, \{G_K\}_{K\in X}, \{A_K\},\inv_{L/K})$ be a class formation. Then there is a isomorphism
\[
\xymatrix{
H_T^{-2}(G(L/K),\Z)\ar[r]^{u_{L/K}\cup \bullet}_{\cong}\ar@{=}[d] & H_T^0(G,A_L)\ar@{=}[d]\\
G(L/K)^\text{ab}& A_K/\nm_{L/K}(A_L)}
\]
\end{thm}
Here $\nm_{L/K}$ means $N_{G_K/G_L}$. Denote the reverse map by $\phi_{L/K}$.
\begin{proof}
The identifications are from Theorem~\ref{h1-is-gab} and Definition~\ref{tate-df}. Axioms 1 and 2 for class formation give that the two conditions of Tate's Theorem~\ref{tate-thm} are satisfied.
\end{proof}
This map is hard to calculate directly because cup products on negative Tate cohomology are hard to deal with. 
The following helps us by transferring the cup products to nonnegative Tate groups.
\begin{thm}\llabel{thm:calculate-local-artin}
Keep the above hypothesis. Then for any $\chi\in \Hom^{\text{cont}}(G(L/K),\Q/\Z)=H^1(G,\Q/\Z)$ and $a\in A_K$,
\[
\chi(\phi_{L/K}(a))=\inv_K(\ol{a}\cup \de \chi).
\]
Here $\ol a$ denotes the image of $a$ in $H_T^{0}(G(L/K),A_L)=A_L/\nm_{L/K}A_L$, and $\de$ is the diagonal morphism corresponding to the exact sequence $0\to \Z\to \Q\to \Q/\Z\to 0$.
\end{thm}
Note this characterizes the reciprocity map since knowing the image of an element of an abelian group under all homomorphisms to $\Q/\Z$ is equivalent to knowing the element itself.\footnote{It may seem odd to calculate $\chi\circ \phi_{L/K}$ instead of $\phi_{L/K}$ directly but keep in mind that for general $L/K$, $\Frob_{L/K}(\mfp)$ is only defined to be a {\it conjugacy class}, and it is natural to look at the action of characters on conjugacy classes because characters are class functions.}
\begin{proof}
%Identify $H_T^{-1}(G,\Q/\Z)\cong \rc n\Z/\Z$, and
Suppose $\chi(\phi_{L/K}(a))=\fc rn$.

By the definition of the Artin map as the inverse of $u_{L/K}\cup \bullet$, we have
\[
\ol{a}
=u_{L/K}\cup \phi_{L/K}(a).
\]
We now calculate the following (for easy reference, we note which cohomology groups the elements are in). 
\begin{align}
\underbrace{\ol a}_{0}\cup \underbrace{\de \chi}_{2}
&=[\underbrace{u_{L/K}}_{2} \cup\underbrace{\phi_{L/K}(a)}_{-2}]\cup \underbrace{\de \chi}_{2}\nonumber\\
\nonumber&=\underbrace{u_{L/K}}_{2} \cup[\underbrace{\phi_{L/K}(a)}_{-2}\cup \underbrace{\de \chi}_{2}]&\text{associativity}\\
\nonumber&=\underbrace{u_{L/K}}_{2} \cup[\de(\underbrace{\phi_{L/K}(a)}_{-2}\cup \underbrace{\chi}_{1})] \nonumber&\text{Theorem~\ref{thm:cup-product}(4)}\\
\nonumber&=\underbrace{u_{L/K}}_{2} \cup \underbrace{\de(\chi(\phi_{L/K}(a)))}_{0}&\text{Theorem \ref{thm:cup-prod-calc}(3)}\\
\nonumber&=\underbrace{u_{L/K}}_{2} \cup \underbrace{\de\pf rn}_{0}\\
\llabel{eq:calc-artin1}&=\underbrace{u_{L/K}}_2\cup \underbrace{r}_{0}\\
\nonumber
&=ru_{L/K}&\text{Theorem \ref{thm:cup-prod-calc}(1)}%\llabel{eq:calc-artin-map2}
\\
\nonumber\inv_K(\ol a\cup \de \chi) &= \fc rn=\chi(\phi_{L/K}(a)).
\end{align}
%In~(\ref{eq:calc-artin-map1})
In~\eqref{eq:calc-artin1}, we use the map in the snake lemma to calculate $\de\pf rn$: it pulls back to $\fc rn\in \Q\cong H_T^{-1}(G,\Q)$; the norm maps it to $r=n\cdot \fc rn\in \Q\cong H_T^{0}(G,\Q)\supeq H_T^{0}(G,\Z)$. In the second-to-last line, %(\ref{eq:calc-artin-map2}), 
we note that $\bullet \cup r$ is simply multiplication by $r$ in dimension 0, so Theorem \ref{thm:cup-prod-calc}(1) tells us it is multiplication by $r$ in dimension $2$ as well.
%We now calculate the following (for easy reference, we note which cohomology groups the elements are in).
%
%Serre, p. 170 (Prop XI.3.2). Cup product calculations...
\end{proof}
We need several naturality properties of the reciprocity map.
\begin{thm}\llabel{thm:reciprocity-natural}
Let $M/L/K$ be Galois extensions. The following are commutative.
\[
\xymatrix{
A_L\ar[r]^{\Cor^0=\nm_{L/K}} \ar[d]_{\phi_{M/L}} & A_K\ar[d]_{\phi_{M/K}} & A_K \ha{r}^{\Res^0=i}\ar[d]_{\phi_{M/K}} & A_L\ar[d]^{\phi_{M/L}}\\
G(M/L)\abe\ar[r]^{\Cor^{-2}}_{\text{natural}} & G(M/K)\abe & G(M/K)\abe\ar[r]^{\Res^{-2}=V} & G(M/L)\abe\\
A_K\ar[d]^{\phi_{L/K}}\ar[r]^{s^*} & A_{sK}\ar[d]^{\phi_{sL/sK}} & A_K\ar[d]^{\phi_{M/K}}\ar[rd]^{\phi_{L/K}}&\\
G(L/K)\abe \ar[r]^{s^*} &G(sL/sK) &
G(M/K)\abe \ar[r] & G(L/K)\abe.
}
\]
\end{thm}
\begin{proof}
First note that the maps in the first diagram are corestrictions and the maps in the second diagram (on the right) are restrictions by Proposition~\ref{pr:extend-to-tate}.

From axiom 4 of Proposition~\ref{def:class-formation}, we have
\[
\Res_{K/L}(u_{M/K})=u_{M/L}.
\]

We will use Proposition~\ref{res-cup}, about the commutativity of cup products with restriction and corestriction. The first diagram follows from 
\[
\Cor^0_{L/K}(x\cup u_{M/L})=\Cor^0_{L/K}(x\cup \Res_{K/L}(u_{M/K}))
=\Cor^2_{L/K}(x)\cup u_{M/K},\quad x\in G(M/L)\abe.
\]
The second diagram follows from
\[
\Res^0_{K/L}(x\cup u_{M/K})=\Res^{-2}_{K/L}(x)\cup u_{M/L}.
\]
The third diagram follows from the fact that the map $s^*:AL\to A_{sK}$ takes $u_{L/K}$ to $u_{sL/sK}$.

For the last diagram, let $\chi$ be a character on $G(L/K)$, which gives a character $\chi_{M/K}$ on $G(M/K)$ using the projection $G(M/K)\to G(L/K)$. By Theorem~\ref{thm:calculate-local-artin} we have, for any character $\chi$,
\[
\chi_{M/K}(\phi_{M/K}(a))=
\inv_K(\ol a_{M/K}\cup \de\chi_{M/K})=\inv_K(\ol a_{L/K}\cup \de\chi)
=\chi(\phi_{L/K}(a))
\]
where $\ol a_{M/K},\ol a_{L/K}$ are the images in $H_T^0(M/K)$ and $H_T^0(L/K)$, respectively. %But this is clear as $\de$ and cup product respect coinflation (need to add this).
\end{proof}
The fourth diagram means that the maps $\phi_{L/K}$ are compatible, so we can define 
\[
\phi_K=\varprojlim_L \phi_{L/K}:
A\to G\abe.
\]
(Note $A=\bigcup A^H$.)
\index{norm limitation theorem}
\begin{thm}[Norm limitation]\llabel{norm-limitation}
Let $(G,\{G_K\},\{A_K\},\inv_{L/K})$ be a class formation. 
Let $L/K$ be an extension and $E/K$ be the largest abelian subextension. Then
\[
\nm_{L/K} A_L=\nm_{E/K} A_{E}.
\]
\end{thm}
\begin{proof}
Let $L\gal$ be the Galois closure of $L$. Transitivity of norms (just look at the definition of norm...) gives us $\subeq$. Conversely, suppose $a\in \nm_{E/K}A_E$. Let $G=G(L\gal/K)$ and $H=G(L'/L)$. Since $E$ is the largest abelian subextension of $L\gal$ abelian over $K$ and contained in $L$, the subgroup of $G$ fixing it is $G'H$. 
We have the commutative diagram
\[
\xymatrix{
A_L \ar[r]^{\phi_{L\gal/L}} \ar[d]^{\nm_{L/K}} & H/H'\ar[d]^i\\
A_K \ar[r]^{\phi_{L\gal/K}} \ar[rd]_{\phi_{E/K}} & G/G'\sj{d}\\
& G/G'H}
\]
where $i$ is induced by inclusion.
Because $a\in \nm_{E/K} A_E$, $\phi_{E/K}(a)=1$ in $G/G'H$. Thus $\phi_{L\gal/K}(a)\in G'H/G'$, and $\phi_{L\gal/K}(a)$ is in the image of $i$ and hence $i\circ \ph_{L'/L}$, and there exists $b\in A_L$ such that $\phi_{L\gal/K}(a)=i(\phi_{L\gal/L}(b))$. Then 
\[
\phi_{L\gal/K}(a)=i(\phi_{L\gal/L}(b))=\phi_{L\gal/K}(\nm_{L/K}(b)).
\]
This means $\fc{a}{\nm_{L/K}(b)}\in \ker(\phi_{L\gal/K})=\nm_{L\gal/K}(A_{L'})$; say it equals $\nm_{L\gal/K}(c)$. Then 
\[
a=\nm_{L/K}(b\nm_{L\gal/L}(c))\in \nm_{L/K}(A_L),
\]
as needed.
\end{proof}
\begin{df}
A subgroup $S$ of $A_K$ is a \textbf{norm group} if there exists an extension $L/K$ such that $S=\nm_{L/K}(A_L)$.
\end{df}
\begin{thm}[Bijective correspondence]\llabel{thm:abstract-bijection}
Let $(G,\{G_K\},\{A_K\},\inv_{L/K})$ be a class formation. Then there is a bijective correspondence between finite abelian extensions of $K$ and the set of norm groups of $A_K$, given by
\[
L\mapsto\nm_{L/K}(A_L).
\]
Furthermore, this is an inclusion-reserving bijection that takes intersections to products and products to intersections:
%\begin{enumerate}
%\item 
%$
\begin{align*}
L\subeq M&\iff \nm_{L/K}(A_L)\supeq \nm_{M/K}(A_M)\\
%$
%\item $
\nm_{L\cdot L'/K}(A_{L\cdot L'})&=\nm_{L/K}(A_L)\cap \nm_{L'/K}(A_{L'})\\%$.
%\item $
\nm_{L\cap L'/K}(A_{L\cap L'})&=\nm_{L/K}(A_L)\cdot \nm_{L'/K}(A_{L'}).
%\end{enumerate}
\end{align*}
Finally, every subgroup of $A_K$ containing a norm group is a norm group. 
\end{thm}
\begin{proof}
Abbreviate $\nm_{L/K}(A_L)$ by $N_L$.

First we show $N_{LL'}=N_L\cap N_{L'}$. %The inclusion ``$\subeq$" is clear from transitivity of norms. Now
By reciprocity,
\[
N_L\cap N_{L'}= \ker(\phi_{L/K})\cap \ker(\phi_{L'/K})\stackrel{(*)}=\ker(\phi_{LL'/K})=N_{LL'}\]
where $(*)$ comes from compatibility of the $\phi$ and the fact that the map $G(LL'/K)\to G(L/K)\times G(L'/K)$ is injective.

If $L\subeq M$, then $N_L\supeq N_M$ from transitivity of norms.
Conversely, if $N_L\supeq N_M$, then by the above $N_L=N_LN_M=N_{LM}$. Thus $[A_K:N_L]=[A_K:N_{LM}]$, and reciprocity gives $[L:K]=[LM:K]$, i.e. $LM=L$, i.e. $L\subeq M$. Thus, $L\mapsto N_L$ is an inclusion-reversing bijection.

Next we show that every subgroup containing a norm group is a norm group. Suppose $N_L\subeq N$; we show $N$ is a norm group. We have that $\phi_{L/K}$ maps $N$ isomorphically onto $G(L/K')$, where $K'=L^{\phi_{L/K}(N)}$, the fixed field of $\phi_{L/K}(N)$. Consider the following commutative diagram from Theorem~\ref{thm:reciprocity-natural}:
%Consider $L^{\phi_{L/K}(N)}$, the fixed field of $\phi_{L/K}(N)$. $\phi_{L/K}$.
\[
\xymatrix{
A_K\ar@{->>}[r]^{\phi_{L/K}} \ar@{->>}[rd]_{\phi_{K'/K}} & G(L/K)\ar[d]\\
&  G(K'/K).
}
\]
From this we find
\[
N=\ker(\phi_{K'/K})=N_{K'}
\]
as needed.

Finally, we show $N_{L\cap L'}=N_L\cdot N_{L'}$. Note $L\cap L'$ is the largest extension contained in both $L$ and $L'$, while $N_L\cdot N_{L'}$ is the smallest group containing both $N_L$ and $N_{L'}$, and it is a norm group by the above. Since $L\mapsto N_L$ is an inclusion-reversing bijection, we must have $N_{L\cap L'}=N_L\cdot N_{L'}$. 
\end{proof}
\subsection{Class formations for local class field theory}
As promised, we apply the results of the last section to $(G(\ol K/K),\ol K)$ where $K$ is a local field. (In the global case we will set $A$ to be the ideles instead.)

\begin{thm}\llabel{thm:lcft-class-form}
Let $L$ be a local field. Then
\[
(G(\ol K/K),\set{G(L/K)}{L/K\text{ finite Galois}},\ol K)\]
is a class formation.
\end{thm}
\begin{proof}
We verify the axioms of class formations. 
\begin{enumerate}
\item $H^1(L/K)=0$ for every cyclic extension of prime degree, by Hilbert's Theorem 90 (\ref{h90}).
\item Take the composition of the isomorphism $H^2(K)\cong H^2(K\ur/K)$ of Theorem~\ref{hkur} with the invariant map $H^2(K\ur/K)\to \Q/\Z$ to get
\[
\inv_K:H^2(K)\to \Q/\Z.
\]
The maps $\inv_{L/K}:H^2(L/K)\hra H^2(K)\to \Q/\Z$ are isomorphisms onto their image, which much be $\rc{[L:K]}\Z/\Z$.

Now we verify that
\[
\inv_L\circ \Res_{K/L}=n\inv_K,\quad n=[L:K].
\]
This follows from the following commutative diagram. From Theorem~\ref{thm:inv-compatible}, the right square commutes; from the fact that inflation commutes with restriction (by functoriality), the left square commutes.
\[
\xymatrix{
H^2(K) \ar[d]^{\Res_{K/L}} & H^2(K\ur/K)\ar[l]^{\cong}_{\Inf}\ar[d]^{\Res_{K/L}} \ar[r]^-{\inv_{K\ur/K}} & \Q/\Z\ar[d]^{n}\\
H^2(L) & H^2(L\ur/L)\ar[l]^{\cong}_{\Inf} \ar[r]^-{\inv_{L\ur/L}} & \Q/\Z.
}
\]
(Note that the target of the restriction in the middle is $H^2(K\ur/L)$, which is a subgroup of $H^2(L\ur/L)$.)
\qedhere
%implicit here is \inv_{L\ur/L}=\inv_{K\ur/K}\Inf_{L\ur/K\ur}
\end{enumerate}
\end{proof}
Applying results about class field theory, we get the main results of local class field theory, restated below.
\begin{thm*}[Local reciprocity law, Theorem~\ref{local-reciprocity}]
For any nonarchimedean local field $K$, there exists a unique homomorphism
\[
\phi_K:K^{\times} \to G(K^{\text{ab}}/K),
\]
called the $\textbf{local Artin (reciprocity) map}$
with the following properties.
\begin{enumerate}
\item (Relationship with Frobenius map)
For any prime element $\pi$ of $K$ and any finite unramified extension $L$ of $K$, $\phi_K(\pi)$ acts on $L$ as $\Frob_{L/K}(\pi)$.
\item (Isomorphism)
Let $p_{L}$ be the projection $G(K\abe/K)\to G(L/K)$. For any finite abelian extension $L/K$, %$\nm_{L/K}(L^{\times}\subeq \ker (p_L\circ \phi_K)$
$\phi_K$ induces an isomorphism $\phi_{L/K}:K^{\times}/\nm_{L/K}(L^{\times})\to G(L/K)$ making the following commute:
\[
\xymatrix{
K^{\times} \ar[r]^{\phi_K}\ar[d] & G(K^{\text{ab}}/K)\ar[d]^{p_L}\\
K^{\times}/\nm_{L/K}(L^{\times})\ar[r]^-{\phi_{L/K}}_-{\cong} & G(L/K).
}
\]
\item (Compatibility with norm map) For any $K\subeq K'$, the following diagram commutes.
\[
\xymatrix{
K'^{\times} \ar[r]^-{\phi_{K'}} \ar[d]^{\nm_{K'/K}} & G({K'}\abe/K')\ar[d]^{\bullet|_{K\abe}}\\
K^{\times} \ar[r]^-{\phi_{K}}& G(K\abe/K)
}
\]
\end{enumerate}
\end{thm*}
\begin{proof}
By Theorem~\ref{thm:lcft-class-form}, $(G(\ol K/K),\set{G(L/K)}{L/K\text{ finite Galois}},\ol K)$ is a class formation. By the Abstract Reciprocity Law applied to $A_K=K$, we thus have an isomorphism $K^{\times}/\nm_{L/K}L^{\times}\xra{\cong} G(L/K)^{\text{ab}}$. These maps are compatible by the first and fourth diagrams in Theorem~\ref{thm:reciprocity-natural}.
% Commutativity of the second diagram comes from Theorem 

Next we show that $\phi_K(\pi)$ acts on $L$ as $\Frob_{L/K}$. For the first, we use Theorem~\ref{thm:calculate-local-artin}, which says
\[
\chi(\phi_{L/K}(\pi))=\inv_K(\ol{\pi}\cup \de \chi).
\]
We calculate the invariant map on $\ol{\pi}\cup \de \chi$, recalling that the map $H^1(G,\Q/\Z)\to \Q/\Z$ is evaluation at the Frobenius:
\[
\xymatrix{
H^2(L/K)\ar[r] & H^2(G,\Z) & H^1(G,\Q/\Z)\ar[l]_{\de} \ar[r] & \Q/\Z\\
\ol{\pi}\cup \de \chi \ar[r] & v(\pi)\cup \de \chi=1\cup \de \chi  & 1\cup \chi \ar[r]\ar[l] & \chi(\Frob_{L/K}).
}
\]
Thus $\chi(\phi_{L/K}(\pi))=\chi(\Frob_{L/K})$ for all characters $\chi$ on $G(L/K)$, and $\phi_{L/K}(\pi)=\Frob_{L/K}$.

We will prove uniqueness in Section~\ref{sec:lcft-uniqueness}
%The first is because the Frobenius map was defined by choosing the Frobenius map as the generator of the cyclic group... (details) 
\end{proof}
%\begin{thm}[Norm limitation theorem]
%Let $L$ be a finite extension of $K$, and $K'$ be the largest abelian extension of $K$ contained in $L$. Then 
%\[
%\nm_{L/K}(L^{\times})=\nm_{K'/K}(K'^{\times}).
%\]
%\end{thm}
\begin{proof}[Proof of norm limitation, Theorem~\ref{thm:lcft-norm-limitation}]
This follows directly from
Theorem~\ref{thm:lcft-class-form} %($(G(\ol K/K),\set{G(L/K)}{L/K\text{ finite Galois}},\ol K)$ is a class formation) 
and Theorem~\ref{norm-limitation}.
\end{proof}
\section{Examples}
Before we move on to the existence theorem, we seek to understand the reciprocity map a bit better. 
\subsection{Unramified case}
The reciprocity map is easiest to understand for unramified extensions.
\begin{ex}\llabel{ex:unramified-rec}
Suppose $L/K$ is an unramified extension of local fields of degree $n$ (possibly infinite). Then the reciprocity map is
\bal
\phi_{L/K}:K^{\times}/\nm_{L/K}(L^{\times})\cong K^{\times}/\pi^{n\Z}U_K&\to G(L/K)\\
a&\mapsto \Frob_{L/K}^{v(a)}.
\end{align*}
\end{ex} 
\begin{proof}
There are many ways to see this. We know that any uniformizer maps to $\Frob_{L/K}$. But the uniformizers generate $K^{\times}$, so $\phi_{L/K}$ must be the map $a\mapsto \Frob_{L/K}^{v(a)}$. As $\Frob_{L/K}$ has order $n$, the kernel is $\pi^{n\Z}U_K$.

Alternatively, in the proof of Theorem~\ref{local-reciprocity} above, run the argument with arbitrary $a$ instead of $\pi$. 
%Alternatively, note $U_K\subeq \nm_{L/K}L^{\times}$ by Corollary~\ref{thm:local-nm-surj}, and the only subgroup of index $n$ containing $U_K$ is $\pi^{n\Z}U_K$.
\end{proof}
\subsection{Ramified case}
To understand the reciprocity map on ramified extensions, we have the following.
\begin{pr}\llabel{pr:unit-to-inertia}
For any Galois extension of local fields $L/K$, 
\[\phi_{L/K}(U_K)\subeq I(L/K),\]
where $I(L/K)$ is the inertia group.
\end{pr}
\begin{proof}
By Theorem~\ref{decomposition-and-inertia}, $L^{I(L/K)}/K$ is the maximal unramified subextension of $L/K$,  so $U_K\subeq \ker(\phi_{L^{I(L/K)}/K})$ from Example~\ref{ex:unramified-rec}. This means that $\phi_{L/K}(U_K)$ projects trivially on $G(L^{I(L/K)}/K)$, i.e. $\phi_{L/K}(U_K)\subeq I(L/K)$.
\end{proof}
In fact, the reciprocity map relates filtration on the unit group $U_K$ with the filtration on ramification groups (cf. Section~\ref{sec:ram-filt}), so Proposition~\ref{pr:unit-to-inertia} is just the beginning of the story.
\begin{thm}
The reciprocity map transforms the filtration
\[
K^{\times}/\nm_{L/K}(L^{\times})\supeq U_K/\nm_{L/K}(U_L)\supeq U_K^{(1)}/\nm_{L/K}(U_L^{\psi(1)})\supeq \cdots
\]
into the filtration
\[
G(L/K)\supeq G^0=I(L/K)\supeq G(L/K)^1\supeq \cdots.
\]
\end{thm}
\begin{proof}
This uses more about local fields and local symbols than we'll prove. See Serre~\cite{Se79}, Chapter XV or Neukirch~\cite{Ne99}, V.\S6.
\end{proof}
\begin{ex}
For the totally ramified extension $\Q_p(\ze_{p^{\iy}})/\Q_p$, the reciprocity map sends
\[
p^{\Z}(1+(p^r))\mapsto G(\Q_p(\ze_{p^{\iy}})/\Q_p(\ze_{p^r})).
\]
The RHS is the $r$th upper ramification group $G^r$.
\end{ex}
Explicit computation of the reciprocity map in the ramified case is difficult without Lubin-Tate Theory.
\section{Hilbert symbols}\llabel{sec:hilbert-symbol}
\index{Hilbert symbol}
To prove the existence theorem, we need to show that every closed subgroup of $G$ occurs as a norm group, i.e. as the kernel of some Artin map $\phi_{L/K}$. To do this, we explicitly construct field extensions $L/K$ that give these norm groups. We will construct Kummer extensions, extensions that come from adjoining an $n$th root. We focus on these extensions for several reasons.
\begin{enumerate}
\item Recall that we don't have a way to directly calculate the action of $\phi_{L/K}$. Instead, we calculate indirectly by Theorem~\ref{thm:calculate-local-artin}: If we know $\chi(\phi_{L/K}(a))$ for all characters on $G(L/K)$, then we have determined $\phi_{L/K}(a)$.

An easy source of characters comes from Kummer Theory~\ref{pr:kummer-char}, since the group of characters is isomorphic to a cyclic group.\footnote{Artin-Schreier theory, from exercise~\ref{galois-cohomology-ch}.2.1, is another source of characters.}
\item We want to show that certain subgroups of norm groups are also norm groups. After verifying several topological properties of $\phi_{K}$, we can reduce this to a statement about $p$th powers/roots of norm groups. In the abstract existence theorem~\ref{thm:abstract-existence}, properties 1 and 3 are easy to check; they are basically the reductions that allow property 2 to be sufficient.
\end{enumerate}
Recally from Proposition~\ref{pr:kummer-char} that $K^{\times}/K^{\times n}\cong \Hom(G(K^s/K),\mu_n)$. Thus the characters we get are in bijection with elements of $K^{\times}/K^{\times n}$. We can also consider $a\in K^{\times}$ as inside $K^{\times}/K^{\times n}$, and this gives us a sort of ``duality": the Kummer pairing. We will see eventually that this is the source of reciprocity laws (Section~\ref{sec:rec-laws}), so these symbols are good for more than just proving the existence theorem.

We assume throughout that $K$ contains a $n$th root of unity, and $\chr(K)\nmid n$.
\begin{df}
Let $G=G(K^s/K)$. 
Define the local symbol 
\begin{align*}(\,,\,)_n:H^1(G,\Q/\Z)\times  \underbrace{H^0(G,K^{s\times})}_{K^{\times}}&\to H^2(G,K^{s\times})=\Br_K\\
(\chi,b)&=\ol b\cup \de \chi
\end{align*}
Here $\de$ is with respect to the exact sequence $0\to \Z\to \Q\to \Z/\Q\to 0$ and $\ol b$ is the image of $K^{s\times}$ in $H_T^0(G,K^{s\times})$.

We will drop the subscript $n$ when the context is clear.
\end{df}
Since cup product is bilinear and $\de$ is linear, $(\,,\,)$ is bilinear.
If $K$ is local, by Theorem~\ref{thm:calculate-local-artin}, we have for any Galois $L/K$ and any character $\chi$ on $G(L/K)$,
\begin{equation}\llabel{eq:invk-calc}
\inv_K(\chi,\phi_{L/K}(a))=\inv_K(a\cup \de \chi)=\chi(\phi_{L/K}(a)).
\end{equation}
%($\inv_K$ is an isomorphism.)

As promised, we now transfer this action to $K^{\times}/K^{\times n}$.
\begin{df}
Suppose $K$ is a local field, and let $G=G(K^s/K)$. %\footnote{If we don't assume this we can still go part of the way though.} 
For $a\in K^{\times}$, define the character as in Proposition~\ref{pr:kummer-char} by 
\[\chi_a(\si)=\fc{\si(a^{\rc n})}{a^{\rc n}},\quad \chi_a\in H^1\pa{G,\rc n\Z/\Z}\cong H^1(G,\mu_{n}),\]
where $G=G(L/K)$ and $L=K(a^{\rc n})$.
Here we choose a root of unity $\ze$ to make a correspondence $\rc n\Z/\Z\cong \mu_n$.

Define the \textbf{Hilbert symbol} by 
\begin{align*}
K^{\times}\times K^{\times}&\to \Br_K[n]\cong \rc n\Z/\Z\cong 
\mu_n\\
(a,b)&:=(\chi_a,b)=b\cup \de \chi_a.
\end{align*}
If $K$ is a global field, let $(a,b)_v$ denote the Hilbert symbol where $a,b$ are considered as elements of $K_v$.
\end{df}
Note that the image is in $\rc n\Z/\Z$, not just in $\Q/\Z$, because $n\chi_a=0$.

We'll abuse notation and not make a clear distinction between $ \Br_K[n]\cong\rc n\Z/\Z\cong \mu_n$, where $\Br_K[n]$ denotes the $n$-torsion subgroup of $\Br_K$. The first isomorphism is given by $\inv_K$ and the second by $\rc n\leftrightarrow \ze$. We transfer the $\chi_a$ from being defined on $\mu_n$ to $\rc n\Z/\Z$, then transfer back from $\Br_K[n]\cong \rc n\Z/\Z$ to $\mu_n$ at the end, so we may as well use the formula~(\ref{eq:invk-calc}) for the $\chi_a$ treated in $H^1(G,\mu_{n})$.

The following relates the Hilbert symbol to the Artin map.
\begin{pr}\llabel{pr:hilbert-explicit}
We have
\[
(a,b)=\fc{[\phi_{L/K}(b)](\sqrt[n]{a})}{\sqrt[n]a}
\]
where $L=K(\sqrt[n]a)$.
\end{pr}
\begin{proof}
Formula~(\ref{eq:invk-calc}) gives (remember we're identifying $ \Br_K\cong\rc n\Z/\Z\cong \mu_n$; by abuse of notation we drop the ``$\inv_K$" because it is an isomorphism)
\[
(a,b)=(\chi_a,\phi_{L/K}(b))=\chi_a(\phi_{L/K}(b))=\fc{[\phi_{L/K}(b)](\sqrt[n]{a})}{\sqrt[n]a}
\]
where $L$ is any field Galois over $K$, containing $\sqrt[n]a$.
\end{proof}
%Note that because we used the isomorphism $\rc n\Z/\Z\cong \mu_n$, we can think of the result $(a,b)$ as in either $\rc\Z/\Z$
\begin{thm}\llabel{thm:hilbert-bilinear}
The Hilbert symbol descends to a nondegenerate skew-symmetric bilinear map
\[
K^{\times}/K^{\times n}\times K^{\times}/K^{\times n}\to \mu_n
\]
satisfying the following.
\begin{enumerate}
\item 
$(a,b)=1$ iff $b\in \nm_{K(a^{\rc n})/K}(K(a^{\rc n})^{\times})$.
\item
If $a\in K^{\times}$, $x\in K^{\times}$, and $x^n-a\ne 0$, then \[(a,x^n-a)=1.\] In particular, 
$(a,-a)=1=(a,1-a)$.
%\item (Skew-symmetry)
%$(a,b)(b,a)=1$
\end{enumerate}
%MENTION in terms of roots of unity.
\end{thm}
\begin{proof}
Everything that went into defining $(,)$ was linear in either variable (cup products, evaluation homomorphisms, snake lemma morphism), so $(,)$ gives a bilinear map $K^{\times}\times K^{\times}\to \mu_n$.

Suppose $\chi$ is an element of order $n$. Then its kernel $\ker(\chi)$ has index $n$ in $G(K^s/K)$. Under the Artin map this corresponds to a extension $L_{\chi}$ of degree $n$, such that $\ker(\chi) = \phi_K(\nm_{L_{\chi}/K} (L_{\chi}^{\times}))$.  
%explain?
Then
\[(\chi,b)=\chi(\phi_K(b))=0\iff \phi_K(b)\in \ker(\chi)\]
iff $b\in \nm_{L_{\chi}/K} (L_{\chi}^{\times})$.

We apply this to $\chi=\chi_a$. Note that $\chi$ has order $[K(a^{\rc n}):K]$ and $\chi_a(G(K^s/K(a^{\rc n})))=0$. Hence $\phi_K(\nm_{K(a^{\rc n})/K}(K(a^{\rc n})^{\times})\subeq \ker\chi_a$. By comparing indices in $G(K^s/K)$, equality holds, giving the first item.

For the second item, note that
\[
x^n-a=\prod_{j=0}^{n-1} (x-\ze_n^j a^{\rc n})
\]
(for any choice of $n$th root). The factors in the product can be grouped into conjugates over $K$, so $x^n-a$ is a norm from $K(a^{\rc n})/K$. Then $(a,x^n-a)=1$ from the first item. Setting $x=0,1$ gives $(a,-a)=1$ and $(a,1-a)=1$.

To show skew-symmetry, note from item 2 and bilinearity that
\[
1=(ab,-ab)=(a,-a)(a,b)(b,a)(b,-b)=(a,b)(b,a).
\]
%so corresponds to $L=K(a^{\rc n})$. (Kummer theory - be more precise...) This gives the first item.

To show nondegeneracy, suppose $b\in K^{\times}$ such that $(a,b)=1$ for all $a\in K^{\times}$; we show $b\in K^{\times n}$. The condition $(a,b)=1$ translates into $\chi_a(\phi_{K}(b))=1$ for all $a$. Now the image of $\phi_K$ is dense in $G(L/K)\abe$ (because it is surjective for every finite extension $L/K$, and $G(L/K)$ has the profinite topology). Hence $\chi_a=0$. This means $a^{\rc n}\in K$, i.e. $a\in K^{\times n}$.
\end{proof}
\begin{cor}\llabel{cor:hilbert-local}
Suppose $K$ is a local field, $K(a^{\rc n})/K$ is unramified, and $b$ is a unit in $K$. Then $(a,b)=1$.

If $K$ is a global field, then $(a,b)_v=1$ in $K_v$ unless either $a$ or $b$ is not a unit in $K_v$, or $K(a^{\rc n})/K$ is ramified (which happen at finitely many places).
%Then $b\in U_K\subeq \nm_{K(a^{\rc n})/K} (K(a^{\rc n})^{\times})$ so $(a,b)
\end{cor}
\begin{proof}
Since $K(a^{\rc n})/K$ is unramified, $U_K\sub \nm_{K(a^{\rc n})/K}(K(a^{\rc n})^{\times})$. The result now follows from Theorem~\ref{thm:hilbert-bilinear}.

The second part says that $(a,b)_v=1$ if $a,b$ are units in $K_v$ and $K(a^{\rc n})/K$ is unramified, which is clear from part 1.
\end{proof}
\begin{rem}
In fact, $(a,b)=i(\chi_a\cup\chi_b)$ where $i:H^2(G,\Z/n\Z)\to \Br_K$. (See Serre, p. 207.) This explains the symmetry better but takes more work to prove. 
%Maybe I'll put it in later.
\end{rem}
%note: action of Artin map trivial when in norm group
\section{Existence theorem}\llabel{sec:local-existence}
\index{existence theorem}
We show that the existence theorem follows from several further (topological) axioms on formations. We then prove that in local class field theory, these axioms are satisfied. 
\subsection{Existence theorem in the abstract}
First, a definition.
\begin{df}
Let $(G, \{G_K\}_{K\in X}, A)$ be a class formation. 
The \textbf{universal norm group} $D_K$ of $K$ is the intersection of all norm groups of $A_K$:
\[
D_K=\bigcap_{L/K} \nm_{L/K}(A_L).
\]
\end{df}
\begin{thm}[Abstract existence]\llabel{thm:abstract-existence}
Suppose that $(G, \{G_K\}_{K\in X}, A)$ is a formation satisfying the following conditions.
\begin{enumerate}
\item
For every extension $L/K$, the norm homomorphism has closed image and compact kernel.
\item
Let $[p]$ denote the map $x\mapsto px$ on $A$. 
For every prime $p$, there exists a field $K_p$ such that for $K$ containing $K_p$, $\ker([p]|_{A_K})$ is compact and $\im([p]|_{A_K})$ contains $D_K$.
\item There exists a compact subgroup $U_K$ of $A_K$ such that every closed subgroup of finite index in $A_K$ containing $U_K$ is a norm group.
\end{enumerate}
Then a subgroup of $A_K$ is a norm group iff it is closed of finite index.
\end{thm}
If the conclusion holds, $nA_K\subeq D_K$ for every $K$, because $nA_K$ is closed of finite index and hence a norm group. Conversely, $D_K\subeq \bigcap_{n\ge 1} nA_K$ because every norm group $N$ has finite index so $n$ kills $A_K/N$ for some $n$. Furthermore, $D_K$ must be divisible: else we could find a norm group $N\supeq D_K$, and $n$ such that $nN\nsupeq D_K$, even though $nN$ is still of finite index. 
(Note we write $A_K$ additively here, but in class field theory, $A_K=K$ and $nA_K$ actually means $A_K^n$.)
The most important condition is item 2, because it will give us these two conditions. This gives us a large set of norm groups, and items 1 and 3 (which are more topological in nature) will give us the rest of the desired norm groups.
%We write $A_K$ additively here, but in local class field theory, $A_K=K$ and $\ph_p$ is the map $x\mapsto x^{\rc p}$. Suppose $S$ is a norm group; it is closed of finite index. Then we can find other norm closed groups of finite index, with intersection $S^{\rc p}$. Thus $S^{\rc p}$ must contain the group of universal norms $D_K$. Theorem 2 is a stronger form of this.
\begin{proof}
\noindent \underline{Step 1:} Suppose axiom 1 holds. We show that for every extension $L/K$, $\nm_{L/K}(D_L)=D_K$.

By transitivity of norms, $\nm_{L/K}(D_L)\subeq D_K$.

Conversely, suppose $a\in D_K$. Since $a\in D_K$, for any extension $M/L$, $A_M$ contains an element $b$ such that $\nm_{M/K}(b)=\nm_{L/K}\nm_{M/L} (b)=a$. Thus \[S_M:=\nm_{L/K}^{-1}(a)\cap \nm_{M/L}(A_M)\]
is nonempty. Since $\nm$ has compact kernel, the first group is compact; since $\nm$ has closed image, the second group is closed; thus $S_M$ is compact. Since the $S_M$ for all $M/L$ form a directed system of compact subsets, $S=\bigcap_M S_M$ is nonempty. Any element of $S$ is an element of $\nm_{L/K}^{-1}(a)\cap D_L$. This shows $a\in \nm_{L/K}(D_L)$.\\

\noindent\underline{Step 2:} Suppose axioms 1 and 2 hold. We show $D_K$ is divisible and
\[
D_K=\bigcap_{n\ge 1} nA_K.
\]

First we show that for every prime $p$, $pD_K=D_K$. Let $L$ be a field containing $K_p$, $a\in D_K$, and set
\[
S_L=[p]^{-1}(a)\cap \nm_{L/K} A_L.
\]
Since $[p]^{-1}(a)$ is compact (as $\ker([p]|_{A_K})$ is compact by axiom 2) and $\nm_{L/K}A_L$ is closed, $S_L$ is compact. Now this set this nonempty: since $a\in D_K=\nm_{L/K}D_L$ by step 1, we can write $a=\nm_{L/K} x$, $x\in D_L$. By axiom 2, $x=py$ with $y\in A_K$, so $b:=\nm_{L/K}y\in S_L$. Then $\bigcap_{L\supeq K_p}S_L$ is nonempty as in step 1. Hence $a\in pD_K$.

This shows $pD_K=D_K$, and we get $D_K=\bigcap_{n\ge 1}nD_K\subeq \bigcap_{n\ge1}nA_K$. 

For the other direction, note that $na$ is the norm of any extension of degree $n$, so $\bigcap_{n\ge 1} nA_K\subeq D_K$.\\

\noindent\underline{Step 3:} Assume all the axioms. We prove the theorem. 

First, note that any norm group is closed by axiom 1, and has finite index by the reciprocity law~\ref{thm:abstract-reciprocity}. Indeed, by transitivity of norm, it suffices to consider Galois extensions, and the reciprocity law says $\nm_{L/K}(A_L)$ has index equal to $G(L/K)\abe$.

Conversely, suppose $S$ is a closed subgroup of finite index $n$. We will find a norm subgroup contained in $S$ and then apply Theorem~\ref{thm:abstract-bijection}. Since $A_K/S$ has order $n$, we get $D_K\subeq nA_K\subeq S$, so 
\[
\bigcap_{N\text{ norm group}} (N\cap U_K)=D_K\cap U_K\subeq S.
\]
Since $N\cap U_K$ are compact ($N$ is closed and $U_K$ is compact) and $S$ is open (closed subgroups of finite index are also open), there exists $N$ such that 
\[
N\cap U_K\subeq S.
\]

Note $U_K+(N\cap S)$ is closed of finite index in $A_K$ because $N,S$ are closed of finite index; we show we can replace $U_K$ with $U_K+(N\cap S)$ above:
\[
N\cap (U_K+(N\cap S))\subeq S.
\]
Suppose $a\in U_K$ and $a'\in N\cap S$ such that $a+a'\in N$. Then $a\in N$, but $N\cap U_K\subeq S$ so $a\in S$ as well. Thus $a+a'\in S$, as needed. 

Now $N\cap (U_K+(N\cap S))$ is is closed of finite index containing $U_K$, so is a norm group by axiom 3. By Theorem~\ref{thm:abstract-bijection}, we get $S$ is also a norm group.
\end{proof}
\subsection{Existence theorem for local class field theory}
\begin{proof}[Proof of Theorem~\ref{local-existence}]
We verify that the class formation for LCFT satisfies the three axioms of Theorem~\ref{thm:abstract-existence}.
\begin{enumerate}
\item
To see that the norm map is closed, note that
\[
\nm_{L/K}(L^{\times})\cap U_K=\nm_{L/K}(U_L) 
\]
because an element is a unit iff its norm is a unit. As $U_L$ is compact and $\nm_{L/K}$ is continuous (Proposition~\ref{pr:nm-cont}), $\nm_{L/K}(U_L)$ is compact and hence closed. Now $\nm_{L/K}(L^{\times})$ is a union of translates of $U_L$, therefore closed as well. 

The kernel of $\nm_{L/K}$ is a closed subset of $U_L$, hence compact.
\item
Take $K_p$ containing all $p$th roots of unity. The kernel of the $p$th power map is the $p$th roots of unity, which is a compact set. Suppose $K\supeq K_p$, and let $b\in D_K$ be a universal norm. Then $(a,b)=1$ for all $a$ by Theorem~\ref{thm:hilbert-bilinear}. Since the $p$th power Hilbert symbol is nondegenerate on $K^{\times}/K^{\times p}$, $a\in K^{\times p}$. Thus $D_K\subeq K^{\times p}$.
\item
Take $U_K$ to be the group of units of $K^{\times}$. The closed subgroups of finite index containing $U_K$ are just $\pi^{n\Z}U_K$ for $n\ne 0$; these are the norm groups of unramified extensions of degree $n$ by Proposition~\ref{ex:unramified-rec}. (Note these extensions exist---just adjoin appropriate roots of unity.)
\qedhere
\end{enumerate}
\end{proof}
\begin{proof}[Proof of Theorem~\ref{lcft-correspondence}]
This follows from Theorem~\ref{thm:abstract-bijection}, Theorem~\ref{thm:lcft-class-form} (class formation for LCFT), and the existence theorem just proved.
\end{proof}
Note the existence theorem gives the following.
\begin{cor}\llabel{cor:univ-norm-1}
The universal norm group $D_K$ is $\{1\}$.
\end{cor}
\begin{proof}
All open subgroups of finite index are norm groups by the Existence Theorem~\ref{local-existence}. The intersection of all open subgroups of finite index is $\{1\}$, as $\bigcap_{m,n} (1+(\pi^m))\pi^{n\Z}=\{1\}$.
\end{proof}
\section{Topology of the local reciprocity map}
We now prove that $\phi_K$ gives a topological isomorphism $K^{\times}\to W(L/K)$.
\begin{proof}[Proof of Theorem~\ref{thm:lcft-topology}]
%We claim that $\phi_{L/K}(U_K)\subeq I(L/K)$, where $I(L/K)$ is the inertia group. Indeed, $L^{I(L/K)}/K$ is the maximal unramified subextension of $L/K$ (Theorem~\ref{decomposition-and-inertia}), so $U_K\subeq \ker(\phi_{L^{I(L/K)}/K})$ (REF). This means that $\phi_{L/K}(U_K)$ projects to trivially on $G(L^{I(L/K)}/K)$, i.e. $\phi_{L/K}(U_K)\subeq I(L/K)$.
%But $\phi_{L^{I(L/K)}/K}$ is the projection of $\phi_{L/K}$, so $\phi_{L/K}(U_K)$ is trivial on $L^{I(L/K)}/K$.
By Proposition~\ref{pr:unit-to-inertia}, $\phi_{L/K}(U_K)\subeq I(L/K)$, so we have the commutative diagram
\[
\xymatrix{
1\ar[r] & U_K \ar[r]\ar[d]^{\phi_{L/K}} & K^{\times}\ar[r]^v\ar[d]^{\phi_{L/K}} & \Z\ar[r]\ar[d] & 1\\
1\ar[r] & I(L/K) \ar[r] & G(L/K) \ar[r] & G(l/k)\ar[r] & 1.
}
\]
where the rightmost vertical map sends 1 to the $p$th power Frobenius ($p=|k|$). The vertical maps factor as
\beq{eq:decomp-K/nmL}
\xymatrix{
1\ar[r] & U_K/\nm_{L/K}(U_L) \ar[r]\ar[d]^{\phi_{L/K}}_{\cong} & K^{\times}/\nm_{L/K}(L^{\times})\ar[r]^-v\ar[d]^{\phi_{L/K}}_{\cong} & \Z/f\Z\ar[r]\ar[d]_{\cong} & 1\\
1\ar[r] & I(L/K) \ar[r] & G(L/K) \ar[r] & G(l/k)\ar[r] & 1.
}
\eeq
where $f=[l:k]$. Recall $\phi_K=\varprojlim_L \phi_{L/K}$.  The intersection of all norm groups is $\{1\}$ by Corollary~\ref{cor:univ-norm-1}, so $\phi_K$ is injective on $K^{\times}$. 

In forming $\phi_K=\varprojlim_L \phi_{L/K}$, we are really considering the embedding
\[
K^{\times}\hra \wh{K^{\times}}:=\varprojlim_L K^{\times}/\nm_{L/K}(L^{\times})\xra{\cong} G(K\abe/K).
\] 
Decomposing $K^{\times}/\nm_{L/K}(L^{\times})$ as in~(\ref{eq:decomp-K/nmL}), we have that 
\begin{enumerate}
\item
$\varprojlim_L U_K/\nm_{L/K}(U_L)\cong U_K$ since $U_K$ is compact, hence complete, so $U_K\cong I(K\abe/K)$. 
\item
$\varprojlim_L \Z/f\Z=\wh{\Z}$.
\end{enumerate}
Thus $K^{\times}\hra \wh{K^{\times}}$ is the embedding $U_K\times \pi^{\Z}\hra U_K\times \pi^{\wh{\Z}}$.

Recalling that $W(L/K)$ is the inverse image of $\Frob^{\Z}\subeq G(\ol k/k)$, we get $\phi_{L/K}:K^{\times}\to W(L/K)$ is a topological isomorphism. In summary, we have the diagram
\[
\xymatrix{
1\ar[r]& U_K\ar[r]\ar[d]^{\phi_K}_{\cong}& K^{\times}\ar[r]\ar[d]^{\phi_K}_{\cong}& \pi^{\Z}\ar[r]\ar[d]^{\cong} &1\\
1\ar[r]& I(K\abe/K)\ar[r]\ar[rd]&W(K\abe/K) \ar[r]\ar@{^(->}[d]& \Frob^{\Z} \ar[r]\ar@{^(->}[d]& 1\\
&&G(K\abe/K)\ar[r] &\Frob^{\wh{\Z}}=G(\ol k/k)\ar[r] &1
}
\]
\end{proof}
\subsection{Uniqueness of the reciprocity map}\llabel{sec:lcft-uniqueness}
Finally, we prove uniqueness. This finishes all the proofs of local class field theory.

We first restate Lemma~\ref{lem:local-uniqueness}.
%
\begin{lem*}
Suppose that $K$ is a nonarchimedean local field, $K\ur$ is the maximal abelian unramified extension of $K$, and $L$ is an abelian extension containing $K\ur$. Let $f:K^{\times}\to G(L/K)$ be a homomorphism satisfying (1) and either (2) or $(2)'$:
\begin{enumerate}
\item
The composition $K^{\times}\xra{f} G(L/K) \to G(K\ur/K)$ takes $\al$ to $\Frob_{K\ur/K}(\pi)^{v(\al)}$.
\item
For any uniformizer $\pi\in K$, $f(\pi)|_{K_{\pi}}=1$, where 
\[
K_{\pi}:=L^{\phi_K(\pi)}.
\]
\item[2'.]
For any finite subextension $K'/K$ of $K_{\pi}$, we have
\[
f(\nm_{K'/K}({K'}^{\times}))|_{K'}=\{1\}.
\]
\end{enumerate}
Then $f$ equals the reciprocity map $\phi_K$.
\end{lem*}
%
\begin{proof}[Proof of Lemma~\ref{lem:local-uniqueness}]
It suffices to prove this for $L=K\abe$. We have the split exact sequence
\beq{eq:lcft-uniq}
1\to U_K^{\times}\to K^{\times}\xra{v} \Z\to 1,
\eeq
where the splitting is determined by the map $\Z\to K^{\times}$ sending $1\mapsto \pi$, and the map $K^{\times}\to U_K$ sending $a\mapsto \fc{a}{\pi^{v(a)}}$.
Under the Artin map,~\eqref{eq:lcft-uniq} gets sent to the split exact sequence of topological groups
\[
1\to I(K\abe/K)=G(K/K\ur)\to W(K\abe/K)\to W(K\ur/K)\cong \Z\to 1
\]
by Theorem~\ref{thm:lcft-topology}.
This gives the exact sequence
\[
1\to G(K\abe/K\ur)\to G(K\abe/K)\to G(K\ur/K)\to 1,
\]
where the splitting is by the map $\Z\cong G(K\ur/K)\to G(K\abe/K)$ sending $1\mapsto \phi_K(\pi)$. This identifies $G(K\abe/K\ur)$ with the quotient group $G(K_{\pi}/K)$ where 
\[K_{\pi}=L^{\ol{\an{\phi_K(\pi)}}}=L^{\phi_K(\pi)}.\]
%and by infinite Galois theory $H\cong G(K'/K)$ and $K\abe=K\ur K'$ for some $K'$. Since $\ol{\an{\phi_K(\pi)}}\subeq G(K\abe/K)$ projects down isomorphically to $G(K\ur/K)$, we can take \[K'=L^{\ol{\an{\phi_K(\pi)}}}=L^{\phi_K(\pi)}=K_{\pi}.\]

If $(2)'$ holds, then for any uniformizer $\pi$, we have that %$\phi(\pi)|_{K_{\pi}}=1$, and hence 
$\pi\in \nm_{K'/K}(K'^{\times})$ for every finite subextension $K'$ of $K_{\pi}$. Then $(2)'$ gives that $f(\pi)|_{K_{\pi}}=1$. Then (2) holds.

We now show if (1) and (2) hold, then $f=\phi$. Indeed, (1) and (2) imply that $\phi(\pi)|_{K\ur K_{\pi}}=f(\pi)|_{K\ur K_{\pi}}$ for any uniformizer $\pi$. But $K\ur K_{\pi}=K\abe$ and the set of uniformizers generate $K^{\times}$ (any unit is the quotient of two uniformizers). Hence $\phi=f$.
%Note that $L=K\ur K_{\pi}$
%For any uniformizer, 
\end{proof}
\begin{proof}[Proof of uniqueness in Theorem~\ref{local-reciprocity}]
Suppose $\phi'$ is another map satisfying the conditions of Theorem~\ref{local-reciprocity}. It suffices to show $\phi'$ satisfies the conditions of Lemma~\ref{lem:local-uniqueness} with $L=K\abe$. By assumption it satisfies (1). For condition $(2)'$, we have $\phi_{K}(\pi)|_{K_{\pi}}=1$ by definition of $K_{\pi}$. Hence $\pi$ is a norm from every finite subextension of $K_{\pi}$. By condition 2 of Theorem~\ref{local-reciprocity}, this shows $\phi'_{K'/K}(\nm_{K'/K}({K'}^{\times}))=\{1\}$ for every subextension $K'/K$ of $L$, as needed. Hence $\phi'=\phi$.
\end{proof}
\section*{Problems}
\begin{enumerate}
\item
Using $\phi_K$, construct a natural bijection between the following two sets.
\begin{itemize}
\item
continuous characters $W(\ol K/K)\to \C^{\times}$ (i.e. continuous representations $W(\ol K/K)\to \GL_1(\C)$).
\item
continuous character $K^{\times}\to \C$ (i.e. continuous homomorphisms $GL_1(K)\to GL(\C)$).
\end{itemize}
This is the ``local Langlands correspondence for $GL_1$ over $K$." Local class field theory generalizes more naturally in this form.
\end{enumerate}
